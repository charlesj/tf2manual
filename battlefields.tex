\section{Battlefields}
Throughout combatant tenure, combatants will encounter many types of battlefields. Generally, combatants should kill those on the other team, while following the teamwork guidelines (see page \pageref{Team_Strategy}). Navigating an unfamiliar map can be quite hard at first, so following fellow combatants can be very helpful.   Combatants should  follow the arrows and Blu and Red signs. Combatants should note the different types of conflict they face upon entering the battlefield:

\subsection{Arena (arena\_*)}
Arena is a heavily round-based mode where combatants fight in a last-man standing scenario. There are no supply cabinets inside of spawn points and a very limited number of health packs. Combatants who die in these battles do not respawn until the beginning of the next battle.   The battle is focused around a center control point that eventually unlocks if the teams take too long to kill each other. Battles in arena matches tend to be much shorter and more intense because of these changes. Examples of maps that use this gamemode are Lumberyard and Ravine.

\subsection{Control Points (cp\_*)}
There are a number of control points throughout the battlefield. Both teams must try to capture and/or hold some or all of the points. Combatants work together to capture these points by occupying the control point and sometimes the area immediately around it, as marked by a black and yellow line. Certain points capture slower than others, but having additional compatants of the same team on the point will increase the rate of capture. The maximum number of combatants on the point that will result in a bonus is four, but note that a Scout counts as two combatants on a point. There are two different types of CP maps, attack/defend maps and five control point push maps. 
  
\subsubsection{Attack/defend}
The RED team begins with control of all the points on the map, the BLU team's goal is to take them all before the time limit runs out, while the RED team is trying to hold the points for the time limit. Each time the BLU team is able to take a point, more time is added. The number of points can vary depending on the map, between 2 and 5. Some of RED's points may begin the game locked and cannot be captured by BLU until the points before them have been captured, but this also depends on the map. Examples of maps that use this gamemode are Dustbowl and Gravelpit.

\subsubsection{Five Control Point Push}
These maps have 5 control points in a linear fashion and are symmetrical across the center point. Each team begins the match by spawning on opposite sides of the map. Both teams are automatically given control of the two points closest to their spawn. Both teams then have to race out to fight over the center point. The team that is able to win the center point then tries to push forward onto the opponent's side of the map and take the other two points (in order), while trying to defend the points they have. Points are locked for capture unless the team owns the point preceding it. The first team to control all 5 points on the map wins. Examples of maps that use this gamemode are Badlands and Well.

\subsection{Capture the Flag (ctf\_*)}
Combatants must bring the intelligence briefcase (a.k.a. intel) to a specific point on the map in order to score points. Capturing the intelligence refers to obtaining the intelligence by touching it to pick it up and bringing it back to the intelligence room.  If the intelligence-carrier is killed by the opposing team, the intel is dropped on the ground at the point they were killed. If untouched by the capturing team, it will return to the original intel room in a number of seconds (time varies by map).  The icon above the dropped intel is a timer that displays the time until this occurs.  If the intel is touched by another teammate, they will pick it up and the timer resets. A player can also willingly drop the intel (default key `L') in order to allow a faster class to pick it up. Arrows on the HUD indicate which direction the intel is in. When a player captures the intelligence, their team recieves ten seconds of critical hits. Both teams have identical bases housing their own top secret briefcase of intel.  The goal is to fight into the enemy's base to take their intel while defending one's own. Examples of maps that use this gamemode are 2Fort and Turbine

\subsection{King of the Hill (koth\_*)}
Capture the central control point and hold it!  Once the control point is held by a team for the required amount of time, it is captured and the team-colored round timer will begin to run down. If either team's timer reaches 0:00 and they have complete control, meaning the other team is not currently nor has recently attempted a capture of the point, that team wins. Examples of maps that use this gamemode include Viaduct and Nucleus. 

\subsection{Payload (pl\_*)}
BLU must escort a large cart full of explosives into the RED base. It is the BLU team's job to push the cart by standing next to it to advance it down the tracks before time runs out. More time can be earned by reaching the 2 or 3 checkpoints on every map. Standing near the cart will also heal and replenish attacking combatant ammunition. The more combatants that stand near the cart, the faster the cart moves, however capture speed does not increase once three people are pushing the cart. The RED team must try to stop the cart at all costs by making sure that BLU cannot push the cart. Just one RED combatant standing near the cart will stop BLU making any progress. If the cart is not moved after 30 seconds of inactivity, the cart will start to slowly move backwards towards the last checkpoint. Examples of maps that use this gamemode are Goldrush and Badwater.

\subsection{Payload Race (plr\_*)}
Payload Race is a variant of the Payload battlefield, where instead of one team defending and the other attacking, both teams have a cart that they must push while also stopping the opposing teams’ cart. There is not a time limit in this mode; the match is won when one team successfully escorts their payload to the destination. An example of a map that uses this gamemode is Pipeline.

\subsection{Territory Control (tc\_*)}
Territory Control consists of several smaller battles over single control points.  The map is broken up into multiple regions.  Each round consists of a battle between two of the regions, one controlled by each team. Individual battles are won by capturing the opposing teams' control point.  Once a team has completely taken over the territory of the other team, a point is awards to it, and the map resets. An example of a map that uses this gamemode is Hydro.

\subsection{Additional Battlefields}
A select number of additional battlefield types are referenced in Appendix F on page \pageref{Additional_Battlefields}.