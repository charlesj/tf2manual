\subsection{Appendix G - Screenshots and Demo Recording}

Team Fortress 2 has the useful ability to record screenshots of the battle as well as record the entire battle for later playback.  The default screenshot key is 'F5.'  Pressing this key will save an image of whatever is on the screen to an image in your TF directory inside of Steam.
\begin{lstlisting}
Program Files\Steam\steamapps\<user name>\Team Fortress 2\tf\screenshots
\end{lstlisting}
Demos record combat for later playback.  They can be used to study one's own combat, or the combat of others.  To record a demo, simply type 'record' into the console.  To name the demo, add it after the command.  Example:
\begin{lstlisting}
record shaningins
\end{lstlisting}
The command to stop recording is simple:
\begin{lstlisting}
stop
\end{lstlisting}
Demos are played back in a similar manner:
\begin{lstlisting}
playdemo <demoname>
\end{lstlisting}
And the command to stop playback:
\begin{lstlisting}
stopdemo
\end{lstlisting}

The interface for controlling playback can be displayed with the command:
\begin{lstlisting}
demoui2
\end{lstlisting}

Note:  Demos are often broken after updates.  That is, the software is unable to playback demos recorded by previous versions of the software.

Demo recordings are stored in the 'tf' directory:
\begin{lstlisting}
Program Files\Steam\steamapps\<user name>\Team Fortress 2\tf\
\end{lstlisting}

There are many demos available of professional and highly skilled players that are available for study.  Demos are also a standard way of combating cheaters.  If a combatant is suspected of cheating, a demo recording of their behavior can be used to verify the accusation.

There are two types of demos: a player demo and a server demo.  A player demo is played back through the player's view.  A server demo is a recording of all players and the viewer is able to 'fly' around the battlefield and view all actions.