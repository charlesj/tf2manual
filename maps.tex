\subsection{Appendix E - Map Installation}
Maps can be installed three ways:
\begin{enumerate}
  \setlength{\itemsep}{1pt}
  \setlength{\parskip}{0pt}
  \setlength{\parsep}{0pt}
  \item Through official Valve updates
  \item Through automated downloads from servers
  \item Manually through various websites which host them
\end{enumerate}


The easiest way to get a map is to search the server browser for a server that is running it (arrange results by map name). Connect to that server, and it will automatically download the map and place it into the correct directory. That map can then be played on the hosting server, or on a new user-created server.  (There are, however, a few maps that can only be played on the hosting servers.)

If it is difficult to find a desired map being played, or a map is taking too long to download, it can downloaded and installed manually. Maps are usually compressed using the RAR or ZIP format and need to be uncompressed first.  The .bsp map file can then be copied into the \textbackslash tf\textbackslash maps directory, found in the steam installation folder:
\begin{lstlisting}
Program Files\Steam\steamapps\<user name>\Team Fortress 2\tf\maps
\end{lstlisting}
Once the .bsp file is placed there, the map can be played.