\subsection{Engineer}
\label{Engineer}
{\bf Bio}:
This amiable, soft-spoken good ol' boy from tiny Bee Cave, Texas loves barbeque, guns, and higher education. Natural curiosity, ten years as a roughneck in the west Texas oilfields, and eleven hard science PhDs have trained him to design, build and repair a variety of deadly contraptions.

{\bf Health}: 125

{\bf Purpose}
Area denial, support for teammates in the field, transportation to front lines

\subsubsection {Weapons}


\begin {center}
\underline {Primary}
\end {center}

{\bf Shotgun}: Same weapon as carried by the Soldier, Pyro, and Heavy except it's the Engineer's main weapon.


\begin {center}
\underline {Secondary}
\end {center}

{\bf Pistol}: A basic pistol, 12 shots per magazine. The Engineer carries a large amount of pistol ammo (over 200 shots!)


\begin {center}
\underline {Melee}
\end {center}

{\bf Wrench}: Also known as "Uhlman Build-Matic Wrench.” The Engineer can use his wrench to upgrade buildings, destroy sappers, and kill Spies(or anyone else for that matter). The wrench has a higher than average base critical hit rate.  In addition, damage dealt by the Engineer's sentry adds towards his damage counter for critical hits.


\begin {center}
\underline {PDA}
\end {center}
{\bf Build Tool}

1.       Sentry Gun (costs 130 metal)

2.       Dispenser (costs 100 metal)

3.       Teleporters (entrance and exit each cost 125 metal)


{\bf Destroy Tool}

 The Engineer can destroy his previously constructed buildings in order to move their gear up or to prevent an enemy from destroying them. A building that has been sapped cannot be destroyed.


\subsubsection {Tactics}
\begin {itemize}

\item All buildings start at level 1 and can be upgraded to level 3.  Each upgrade level costs 200 metal and adds more health and increased abilities.

\item Constructed buildings and those of friendly Engineers can be repaired, upgraded, and stocked with ammo by whacking them with the Wrench.  All repairs and upgrades cost metal, and once metal supply is depleted buildings cannot be upgraded or repaired unless more metal is obtained from an ammuniton source.  (See 'Maintaining Ammunition' on page \pageref{Maintaining_Ammunition}). Teleporters share metal towards an upgrade, so fully upgrading both teleporter sides to level 3 costs only 400 metal.

\item An Engineer hitting a building with his wrench while it is being built makes it build twice as fast. Multiple Engineers make this effect stack.

\item Engineers' buildings can be rotated with the alt-fire button before the building is placed in the Build Menu. By utilizing this feature, teleporter exits can be pointed away from walls (note the small arrow on the blueprint of the teleporter exit). This allows teammates who take the teleporter to quickly join the battle, and reduces disorientation.
\end {itemize}
{\bf HUD}:
The building status display (top left corner): Has four slots, one for each building, and shows the status of the buildings that have been built. If a building has been damaged, the respective icon will  flash and display a picture of a Wrench, as well as playing a sound to warn the player. If a Spy's Sapper (see page \pageref{sapper}) has been attached to a building, the same will happen but will be changed with the picture of a Sapper instead of a Wrench.
 
Note: The Engineer is the only class not yet to receive an update.